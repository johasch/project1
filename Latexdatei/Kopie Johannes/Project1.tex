\documentclass[10pt,a4paper]{article}
\usepackage[utf8]{inputenc}
\usepackage[english]{babel}
\usepackage[T1]{fontenc}
\usepackage{amsmath}
\usepackage{amsfonts}
\usepackage{amssymb}
\usepackage{makeidx}
\usepackage{graphicx}
\usepackage{fourier}
\usepackage[left=2cm,right=2cm,top=2cm,bottom=2cm]{geometry}
\author{Johannes Scheller, Vincent Noculak Lukas Powalla}
\title{Computational Physics - Project 1}
\begin{document}
\maketitle
\newpage
\tableofcontents
\newpage
\section{Introduction to Project 1}
In physics, we often have to deal with differential equations o second order, which can be generally written in the from
\begin{equation}
	\frac{d^{2}y}{dx1{2}}+k^2(x)y=f(x)\quad,
\end{equation}
where we call $f$ the inhomgeneous term and $k^2(x)$ is a real function. A special case of these cases is Poisson's equation, which reads in the one-dimensional, spherical case
\begin{equation}
	\frac{1}{r^2}\frac{d}{dr}\left(r^2\frac{d\Phi}{dr}\right)=-4\pi\rho\left(\vec{r}\right)\quad.
\end{equation}
Doing some substitutions, we can write this in the following, more general form:
\begin{equation}
	\label{diff}
	-u''(x)=f(x)
\end{equation}
In this project, we try to solve eq.\eqref{diff} with the boundary conditions $u(0)=u(1)=0$. Therefore, we have to discretize $f$ and $u$. We approximate $u(x)$ as $v_i$, using a grid of $n$ gridpoints $x_i=i\cdot h$. Thus, $h=1/(n+1)$ is our steplength. We will also write $f_i=f(x_i)=f(hi)$. Approximating the second derivative of $u$, we get
\begin{equation}
	\label{num}
	-\frac{v_{i+1}+v_{i-1}-2v_i}{h^2}=f_i
\end{equation}
Our goal is to solve this equation \eqref{num}.



%
%\begin{equation}
%    {\bf A} = \left(\begin{array}{cccccc}
%                           b_1& c_1 & 0 &\dots   & \dots &\dots \\
%                           a_2 & b_2 & c_2 &\dots &\dots &\dots \\
%                           & a_3 & b_3 & c_3 & \dots & \dots \\
%                           & \dots   & \dots &\dots   &\dots & \dots \\
%                           &   &  &a_{n-2}  &b_{n-1}& c_{n-1} \\
%                           &    &  &   &a_n & b_n \\
%                      \end{array} \right)\left(\begin{array}{c}
%                           v_1\\
%                           v_2\\
%                           \dots \\
%                          \dots  \\
%                          \dots \\
%                           v_n\\
%                      \end{array} \right)
%  =\left(\begin{array}{c}
%                           \tilde{b}_1\\
%                           \tilde{b}_2\\
%                           \dots \\
%                           \dots \\
%                          \dots \\
%                           \tilde{b}_n\\
%                      \end{array} \right).
%\end{equation}

\subsection{Rewrite the set of equations in matrix-form}
Eq \eqref{num} can be written as a set of linear equations in matrix form. Therefore, we have to do the following steps:
\begin{align}
	-\frac{v_{i+1}+v_{i-1}-2v_i}{h^2}&=f_i\nonumber\\
	-\left(v_{i+1}+v_{i-1}-2v_i\right)&=h^2\cdot f_i\\
\end{align}

\begin{equation}
{\bf A} = \left(\begin{array}{cccccc}
2& -1& 0 &\dots   & \dots &0 \\
-1 & 2 & -1 &0 &\dots &\dots \\
0&-1 &2 & -1 & 0 & \dots \\\dots
& \dots   & \dots &\dots   &\dots & \dots \\
0&\dots   &\dots  &-1 &2& -1 \\
0&\dots    &\dots  & 0  &-1 & 2 \\
\end{array} \right)
\end{equation}
\end{document}