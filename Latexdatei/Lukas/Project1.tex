\documentclass[10pt,a4paper]{article}
\usepackage[utf8]{inputenc}
\usepackage[english]{babel}
\usepackage[T1]{fontenc}
\usepackage{amsmath}
\usepackage{amsfonts}
\usepackage{amssymb}
\usepackage{makeidx}
\usepackage{graphicx}
\usepackage{fourier}
\usepackage{listings}
\usepackage{color}
\usepackage[left=2cm,right=2cm,top=2cm,bottom=2cm]{geometry}
\author{Johannes Scheller, Vincent Noculak Lukas Powalla}
\title{Computational Physics - Project 1}
\begin{document}
\lstset{language=C++,
	keywordstyle=\bfseries\color{blue},
	commentstyle=\itshape\color{red},
	stringstyle=\color{green},
	frame=single}
\maketitle
\newpage
\tableofcontents
\newpage
\section{Introduction to Project 1}
In physics, we often have to deal with differential equations o second order, which can be generally written in the from
\begin{equation}
	\frac{d^{2}y}{dx1{2}}+k^2(x)y=f(x)\quad,
\end{equation}
where we call $f$ the inhomgeneous term and $k^2(x)$ is a real function. A special case of these cases is Poisson's equation, which reads in the one-dimensional, spherical case
\begin{equation}
	\frac{1}{r^2}\frac{d}{dr}\left(r^2\frac{d\Phi}{dr}\right)=-4\pi\rho\left(\mathbf{r}\right)\quad.
\end{equation}
Doing some substitutions, we can write this in the following, more general form:
\begin{equation}
	\label{diff}
	-u''(x)=f(x)
\end{equation}
In this project, we try to solve eq.\eqref{diff} with the boundary conditions $u(0)=u(1)=0$. Therefore, we have to discretize $f$ and $u$. We approximate $u(x)$ as $v_i$, using a grid of $n$ gridpoints $x_i=i\cdot h$. Thus, $h=1/(n+1)$ is our steplength. We will also write $f_i=f(x_i)=f(hi)$.\\Approximating the second derivative of $u$, we get
\begin{equation}
	\label{num}
	-\frac{v_{i+1}+v_{i-1}-2v_i}{h^2}=f_i
\end{equation}
Our goal is to solve this equation \eqref{num}. Therefore, we will rewrite it as a set of linear equations in matrix form.

\subsection{Rewriting the equation in matrix-form}
Eq \eqref{num} can be written as a set of linear equations in matrix form. Therefore, we have to do the following steps:
\begin{align}
	-\frac{v_{i+1}+v_{i-1}-2v_i}{h^2}&=f_i\nonumber\\
	\label{vorform}-\left(v_{i+1}+v_{i-1}-2v_i\right)&=h^2\cdot f_i\\
\end{align}
Assuming we have an $n\times n$-matrix $\mathbf A$ of the following form

\begin{equation}
{\bf A} = \left(\begin{array}{cccccc}
2& -1& 0 &\dots   & \dots &0 \\
-1 & 2 & -1 &0 &\dots &\dots \\
0&-1 &2 & -1 & 0 & \dots \\\dots
& \dots   & \dots &\dots   &\dots & \dots \\
0&\dots   &\dots  &-1 &2& -1 \\
0&\dots    &\dots  & 0  &-1 & 2 \\
\end{array} \right)\quad,
\end{equation}
then we can rewrite this matrix in index notation as
\begin{equation}
\label{index}
	a_{ij}=\begin{cases}
		2&\mbox{if }i=j\\
		-1&\mbox{if  }|i-j|=1\\
		0&\mbox{else}
	\end{cases}\quad.
\end{equation}
Using this, we can also rewrite the multiplication $\mathbf{w}=\mathbf{A}\cdot\mathbf{v}$ with an $n$-dimensional vector $\mathbf v$ in the following way:
\begin{equation}
	w_i\quad\underset{\mathrm{def}}{=}\quad\sum_{j=1}^{n}a_{ij}\cdot v_j\quad\underset{\eqref{index}}{=}\quad -v_{i-1}+2v_i-v_{i+1}
\end{equation}
By using this result and substituting $h^2\cdot f_i\rightarrow\bar{b}$ in eq \eqref{vorform}, we get to the following equation:
\begin{equation}
	\mathbf{A \cdot v}=\bar{\mathbf{b}}
\end{equation}
with matrix $\mathbf{A}$ as given above. This is a set of linear equations that we are going to solve with our programm. In this example, we will assume that $f(x)$ is given by $f(x)=100\mathrm{e}^{-10x}$. Thus, the analytical solution of eq \eqref{diff} is given by $u(x)=1-(1-\mathrm{e}^{-10})x-\mathrm{e}^{-10x}$. We will later compare our numerical results to this solution.

\section{Different algorithm to solve this set of linear equations}

In this chapter, We want to solve the given set of linear equations first by solving it with a self-programmed algorhithm. Furthermore, we want to compare the produced numerical solution to the exact solution and calculate the maximum relative error for a given number of Gridpoints. 
In order to know, whether our algorithm is best to solve this set of linear equations, we want to compare this algorhith with LU-decomposition and Gaussian algorhitm with respect to the floating point operations and the calculation time. 

\subsection{self-programmed algorithm }

As we have shown in the previous capture, you can rewrite the set of linear equations as product of a matrix and a vector:
\begin{equation}
	\mathbf{A \cdot v}=\bar{\mathbf{b}}
\end{equation}
In this case, the matrix A is a tridiagonal matrix. This means, that it is possible to interpret this matrix as 3 vectors because all other components of the matrix are zero and will stay zero. This gives us the advantage that we don't have to deal with 2-dimensional Arrays and we can dump the number of floating operations (compared to LU-decomposition/Gaussian-algorithm)

The programmed algorithm will work in the following way. First, we want to reach a upper diagonal matrix by vector-additions/subtractions. Second, We want to bring the matrix to a diagonal form and at last, we scale the solution vector. The first two steps also effekt the solutionvector (in our case btilde[] ). We programmed following souce-code (extract):

extract of the used algorithm((12 floating operations):
\begin{lstlisting}
//first
for(int i=0;i<n+1;i++){
	b[i+1]=b[i+1]-c[i]*(a[i+1]/b[i]);
	btilde[i+1]+=-(a[i+1]/b[i])*btilde[i];
}
//second
for(int i=n;i>0;i--){
	btilde[i-1]+=-btilde[i]*(c[i-1]/b[i]);
}
//normalization
for(int i=0;i<n+1;i++){
	btilde[i]=btilde[i]/b[i];
}
\end{lstlisting}
	
	
As you can see, we use at the moment 12n floating point operations for the mainalgorithm. If we look closer at the algorithm, we will see that you can simplify a lot by skipping unnecessary arithmetic operations. 
We managed to get to 6 Floating point operations:
\begin{lstlisting}
//first
for(int i=0;i<n+1;i++){
	b[i+1]=b[i+1]-1/b[i];
	btilde[i+1]=btilde[i+1]+btilde[i]/b[i];
}
//second
for(int i=n;i>0;i--){
	btilde[i-1]=(btilde[i-1]+btilde[i])/b[i];	
}
\end{lstlisting}
	
\subsection{Gaussian elemination}


\subsection{LU-decomposition}

\section{Resuluts and discussion of the self-programmed algorithm}

discussion... errors...plots... gerneral things about why things appear.. 

\section{Comparison of self-programmed algorithm, Gaussian elemination and LU Decomposition }




\end{document}